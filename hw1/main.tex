\documentclass[12pt]{article}
\usepackage[utf8]{inputenc}
\usepackage{amsmath}
\usepackage{amssymb}
\usepackage{graphicx}

\graphicspath{ {./plots/} }

\newcommand{\rectres}[1]{
\begin{center}
\begin{tabular}{ |c| }
\hline\\
#1\\
\\
\hline
\end{tabular}
\end{center}
}

\newcommand{\qed}{\hfill$\blacksquare$}

\title{Introduction to AI\\HW1}
\author{Yair Nahum 034462796\\and\\Hala 209419134 }

\begin{document}

\maketitle

%\tableofcontents{}

\section{Intro}

\subsection{}
We use the notation as in the tutorial about the sets as follows:\\
$$[n] \equiv \{ 1,2,...,n\}$$
Next, we define the $\{S,O,I,G\}$ as follows:\\
For the states we know we have 25 places for the taxi to be in the grid.\\
We have 5 states for the passenger to be $\{R,Y,G,B\}\cup\{in the taxi\}$.
and 4 states for the destination $\{R,Y,G,B\}$.\\
Thus, 
$$S = [500] \Rightarrow $$
$$ |S| = 25*5*4 = 500$$
The valid states as defined in the hw notebook are a subset of this states space as the passenger and destination are not at the same location (3 options are left to seleect from for the destination). So,\\
$$ |S| = 25*5*3 = 375$$
The operations/actions that our taxi agent can perform are:\\
$$O = \{0="South",1="North",2="West",3="East",4="Pickup",5="Dropoff"\}\Rightarrow$$
$$|O| = 6$$
The initial states can be random, but as defined in the hw notebook, we always starts at $s = 328$. Thus,\\
$$I=\{328\}\Rightarrow |I|=1$$
The goal state is when the taxi and passenger are at the destination after the passenger was dropped there. So we have 4 different goal states:\\

$$G=\{s\in S | \{Taxi\wedge Passenger \wedge Destination\in R\} \cup $$
$$\{Taxi\wedge Passenger \wedge Destination\in Y\} \cup $$
$$\{Taxi\wedge Passenger \wedge Destination\in G\} \cup $$
$$\{Taxi\wedge Passenger \wedge Destination\in B\} \} \Rightarrow$$
$$|G|=4$$
\subsection{}


\end{document}

