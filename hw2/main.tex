\documentclass[12pt]{article}
\usepackage[utf8]{inputenc}
\usepackage{amsmath}
\usepackage{amssymb}
\usepackage{graphicx}
\usepackage{algpseudocode}
\usepackage{comment}

\graphicspath{ {./} }

\newcommand{\rectres}[1]{
\begin{center}
\begin{tabular}{ |c| }
\hline\\
#1\\
\\
\hline
\end{tabular}
\end{center}
}

\newcommand{\qed}{\hfill$\blacksquare$}

\title{Introduction to AI - 236501\\HW2}
\author{Yair Nahum 034462796\\and\\Hala Awwad 209419134 }

\begin{document}

\maketitle

%\tableofcontents{}

\section*{A - Improved Greedy Agent}

\subsection*{A.1}

We will use the following parameters:\\
1. Agents' locations.\\
2. Passengers' locations.\\
3. Agents' cash.\\
4. Agents' fuel.\\
5. Gas stations locations.\\

Our agent difference in cash is playing a part in our value function heuristic as it gives us some indication
that our cash was increased (in case we drop a passenger in its destination) at this turn relative to
other operations we may take. We denote it by:
$$cash\_value = taxi.cash - other\_taxi.cash$$

Another important ingredient to our total value function is the fuel. Fuel worth money and the ability to get
more cash by serving passengers. It also plays a major part in our survivable in the game. If we don't refill
the fuel when it's close to empty, we may get stuck. We denote it by:
$$fuel\_value = taxi.fuel - other\_taxi.fuel$$

The Agents' locations and Passengers' locations play a major role in planning the agent's strategy and go towards almost assured cash gain vs the other agent's possibilities.
We calculate for each the possibilities:\\

if both passengers are waiting in the street we calculate the MD to the P0,P1 from each T0,T1. We also calculate the gain we can get after we drop the passenger. for example for T0->P0->D0:
$$taxi\_to\_dest\_value = 2 * MD(P0,D0)- (MD(T0,P0) + MD(P0,D0)) =$$
$$ MD(P0,D0)- MD(T0,P0)$$
We then compare the values from each taxi to each destination.
If we have both feasible before the other agent may reach the passenger, we select the one with the higher value. If only one of them is feasible before the other agent we go towards that passenger (we take that value).\\
If both are not feasible to take before the other taxi, we still select the one with the bigger value.\\

In case we already have a passenger in the taxi, the value is calculated as the cash at the end with the subtraction of the remaining cost:\\
$$taxi\_to\_dest\_value = 2 * MD(P0,D0)- (MD(T0,P0) + MD(P0,D0)) =$$
$$ 2 * MD(P0,D0)- (0 + MD(T0,D0)) = 2 * MD(P0,D0) - MD(T0,D0)$$

In case, we don't have a passenger but the other taxi does have, we calculate the value of getting the other passenger:
$$taxi\_to\_dest\_value = MD(P0,D0)- MD(T0,P0)$$

When the fuel gets below 7 (as the max MD on the board is 6), we add another value function that favors the gas stations. We calculate the MD towards each of it and get the max between the two.\\
We then calculate the value to survive and give it a weight of 10 as it's important not to get stuck w/o fuel:\\
$$max\_cost\_taxi\_to\_gas = max(cost\_taxi\_to\_gas0, cost\_taxi\_to\_gas1)$$
$$survival\_value = 10*(6 - max\_cost\_taxi\_to\_gas)$$

The total value , for each state we get to, is an accumulation of the above:
\rectres{$total\_value = survival\_value + taxi\_to\_dest\_value + cash\_value + fuel\_value$}


Important note, if we discover we can win in certain conditions the other agent, for example, the other taxi has no fuel and we have cash, or the other agent cannot gain enough cash to win our taxi, we don't add the survival value (filling fuel) and just keep wasting fuel until the game is done.

\subsection*{A.2}

The existing heuristic of the greedy agent is not performing well, as it gives for each next state the same value w/o considering the ability to gain more cash due to serving passengers.\\
The only state in which it directs us to the correct direction to gain cash is when it got a passenger and it arrives to its destination. In such case, it will drop it correctly and gain cash as the value increases compared to other next state values which have less cash gain.\\
We've explained in detail the formula and the motivation on the previous section.\\
We think and can see from running some tests that our improved greedy agent wins the original one almost always unless we hit corner cases (such as the other agent gets stuck with one passenger on the other passenger location or destination, thus we cannot take the remaining passenger)\\

\subsection*{A.3}
We've run games with max of 50 steps each.\\
We've run the following with random seeds increment between 0 to 50:\\
greedy vs greedy\_improved:\\
taxi greedy won 0/100 games!\\
taxi greedy\_improved won 58/100 games!\\
\\
greedy vs random:\\
taxi greedy won 0/100 games!\\
taxi random won 4/100 games!\\
\\
greedy\_improved vs random:\\
taxi greedy\_improved won 68/100 games!\\
taxi random won 1/100 games!\\
\\
We clearly see that the improved greedy wins much more.\\
There are corner cases in which it gets a draw such as when the other taxi agent gets one passenger and gets stuck w/o fuel on the other passenger destination.

\begin{comment}
 
\end{comment}
\end{document}

